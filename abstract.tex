In July 2012, a Higgs-like boson was observed jointly at CMS and ATLAS at CERN's Large Hadron Collider. For this thesis, we will revisit the theoretical motivation of the Higgs boson in the Standard Model, including its expected properties of production and decay. Using the $H\rightarrow ZZ\rightarrow 4l$ decay channel on the first run of CMS data in 2011 and 2012, we will establish the procedure used to observe the Higgs boson at its current statistical significance at $\sim7\sigma$. The first measurements of the boson's mass ($m_{H}=125.6$ $\rm{GeV}$), signal strengths ($\mu_F = 0.80^{+0.46}_{-0.36}$, $\mu_V=1.7^{+2.2}_{-2.1}$), width ($\Gamma_{H}<46$ $\rm{MeV}$), and spin-parity ($J^{PC} = 0^{++}$) will be discussed along with exclusions on additional Higgs-like bosons in the $H\rightarrow VV$ decay. Using the kinematics of the decay and production, all measured properties of the observed Higgs boson will be shown to agree within uncertainty with Standard Model predictions. Finally, the sensitivities of future Higgs boson property measurements will be discussed and quantified for the lifetime of the LHC and proposed future colliders, where the projections are comparable to some Beyond the Standard Model predictions. 

\vspace{1cm}

\noindent Primary Reader: Andrei Gritsan\\
Secondary Reader: Morris Swartz